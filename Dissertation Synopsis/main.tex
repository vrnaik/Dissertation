\documentclass[a4paper, 12pt, notitlepage]{article}
\usepackage[a4paper, total={6in, 8in}]{geometry}
\usepackage[]{graphicx}
%	\usepackage{dirtree}
%	\usepackage{enumitem}
%	\usepackage[nottoc]{tocbibind}
%	
%	\usepackage{listings}
%	\usepackage[]{hyperref}
%	\usepackage{tikz}
%	\usetikzlibrary{arrows,shapes,snakes,automata,backgrounds,petri}
%	\usepackage{tabularx}
%	\usepackage{nopageno}
%	\usepackage{lipsum}
%	\usepackage{minted}
%\usepackage{multirow}
%\usepackage{enumitem}
%\usepackage[toc,page]{appendix}
%\usepackage{hyperref}
%\usepackage[]{algorithm2e}
%\usepackage{tikz}
%\usepackage{dirtree}
%\usetikzlibrary{backgrounds}
%\usepackage{xcolor}
%\usepackage{caption} 
%\captionsetup[table]{skip=5pt}
%\newcommand\tab[1][1cm]{\hspace*{#1}}
\begin{document}
\begin{titlepage}
\begin{center}
\textbf {A SYNOPSIS ON}
\linebreak 
\linebreak 
\textbf {\Large{THE RELIABILITY OF HETEROGENEOUS SENSOR NODE CONNECTIVITY IN IoT }}
\linebreak
\linebreak
SUBMITTED TO THE SAVITRIBAI PHULE PUNE UNIVERSITY, 
PUNE IN PARTIAL FULFILLMENT OF THE REQUIREMENTS 
FOR THE AWARD OF THE DEGREE
\linebreak
\linebreak
\textbf{\large{MASTER OF ENGINEERING (Computer Engineering)}}
\linebreak
\textbf{\large{BY}}
\linebreak
\large{Name Mr. Vikas R. Naik  \hspace{10mm}   Exam No. 6109}
%\textbf{\large{Sponsored by}}
%\linebreak
%\large{*** Company Name (Sentence Case) ***}
%\linebreak
\linebreak
\linebreak
\textbf{\large{Under the guidance of}}
\linebreak
\Large{*** External Guide ***     \hspace{10mm}     Prof. Sarang Joshi}
\linebreak
\begin{figure}[ht!]
\begin{center}
\includegraphics[scale=0.3]{pict.png}
\end{center}
\label{overflow}
\end{figure}
\textbf{\linebreak \large{DEPARTMENT OF COMPUTER ENGINEERING}}
\linebreak
\textbf{Society for Computer Technology and Research}\\
\textbf{PUNE INSTITUTE OF COMPUTER TECHNOLOGY}\\
\textbf{PUNE-411043} \\
\end{center}
\end{titlepage}
\pagebreak

\section{INTRODUCTION}

The IoT configured system controls the charging of the consumable data resource of varying rate of data accumulation. These resources are located geographically isolated locations. There is need of quality assurance and reliability foe the availability of the data for the consumption based on demand. There is need of tracking the performance and control the rate of flow of data. Each IoT system has a local operational plan organized to execute using existing machine interface connected to heterogeneous devices having geological constraints, functioning and categorized as essential equipment and other lower priority equipment. It is the necessity to develop the system to supervise the equipment for optimum consumption of the green energy for better service of the system as per the scheduled tasks.

\section{Technical Keywords (ACM keywords):}

\large{IoT, Big Data, Data Science, Data Analysis}

\section{MOTIVATION}

This dissertation focus on the effective control of inflow and outflow data of consumable resource. depending upon the different sunlight exposure with direction of solar panel and the sunlight intensity optimal solution is formulated to maximize the reliability of heterogenous nodes and consumable resource. initially the feasible expression is formulated to control the flow of data and then the experiment is setup for the analysis and performance measurement.


\section{LITERATURE SURVEY}



\section{PROBLEM DEFINITION}

To increase the reliability of heterogeneous node connectivity in IoT  environment for data accumulation and consumption.
\begin{description}
  %\item[$\cdot$ bla1] item 1
  %\item[$\bullet$ bla2] item 2
  %\item[$\ast$ bla3] item 3
\item[$\bullet$ 1]  To test the performance of producer consumer problem in IoT enable sensor inflow-outflow management.
\item[$\bullet$ 2] To test reliability of the sensor in the IoT based sensor network.
\item[$\bullet$ 3] To test the feasibility of optimal usage of the consumable resource.

\end{description}

\section{OBJECTIVES}

\begin{description}
  %\item[$\cdot$ bla1] item 1
  %\item[$\bullet$ bla2] item 2
  %\item[$\ast$ bla3] item 3
\item[$\bullet$ 1] To develop algorithm for effective sensor data inflow and outflow performance.
\item[$\bullet$ 2] To analyze the performance and optimize sensor data inflow an outflow.
\item[$\bullet$ 3] To implement the setup using state of the art and improve the reliability of the system


\end{description}



%==========================================================================
%\section{METHODOLOGY}
%
%
%%==========================================================================
%
%\section{RESULT AND DISCUSSIONS}


%==========================================================================
\section{CONCLUSION}
%
%\section{PAPERS PUBLISHED}
%
%\subsection{PAPER TITLE : ***************}
%    
%    \begin{itemize}
%        \item Fifth Post Graduate Conference of Computer Engineering cPGCON 2016, March 25-26, 2016
%    \end{itemize}
%    
%\subsection{PAPER TITLE}
%    \begin{itemize}
%        \item **********************
%    \end{itemize}
%
%\subsection{PAPER TITLE :*****************}
%    
%    \begin{itemize}
%        \item ************
%        \item Paper Status : **************
%    \end{itemize}
    

%\begin{thebibliography}{1}
%
%        \bibitem{paper1} %
%        Zahra Jamshidi and Farshad Khunjush 
%        \emph{Optimization of OpenFOAM’s Linear Solvers on Emerging Multi-core Platforms}
%        \hskip 1em plus           0.5em minus 0.4em \relax School of Electrical and Computer Engineering Shiraz University, Shiraz, Iran,  Vol 6, No 5, 2011, p824-829
%
%        
%\end{thebibliography}

\vspace{20mm}

\begin{flushleft}

\textbf{Mr. Vikas R. Naik}    \hspace{40mm}       \textbf{Prof. Sarang Joshi}
\linebreak
ME ( Computer Engineering ) \hspace{30mm}   Internal Guide
\end{flushleft}


\end{document}